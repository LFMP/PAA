%This document uses PGF 3.0 graph package.
%Compile it with lualatex.

\documentclass[a4paper]{exam}

\usepackage[utf8]{inputenc}
\usepackage{fontspec}
\usepackage[portuguese]{babel}
\usepackage{amsmath,amssymb, alltt}
\usepackage[margin=1.5cm,nohead,nofoot]{geometry}
\usepackage{multicol}
\usepackage{mathtools}
\usepackage{tikz}
\usepackage{graphicx}
\usepackage{caption}
\usepackage{subcaption}
\usepackage{float}  
\usepackage{color}
\usepackage[T1]{fontenc}
\usepackage{verbatim}

\usetikzlibrary{arrows,calc, snakes, backgrounds,positioning}

\tikzstyle{intt}=[draw,text centered,minimum size=6em,text width=5.25cm,text height=0.34cm]
\tikzstyle{intl}=[draw,text centered,minimum size=2em,text width=2.75cm,text height=0.34cm]
\tikzstyle{int}=[draw,minimum size=2.5em,text centered,text width=3.5cm]
\tikzstyle{intg}=[draw,minimum size=3em,text centered,text width=6.cm]
\tikzstyle{sum}=[draw,shape=circle,inner sep=2pt,text centered,node distance=3.5cm]
\tikzstyle{summ}=[drawshape=circle,inner sep=4pt,text centered,node distance=3.cm]


\usepackage[ruled,vlined,noend,linesnumbered]{algorithm2e}
\setlength{\algoheightrule}{0pt}
\setlength{\algotitleheightrule}{0pt}




\DeclarePairedDelimiter{\ceil}{\lceil}{\rceil}
\DeclarePairedDelimiter{\floor}{\lfloor}{\rfloor}

\footer{}{}{}

\checkboxchar{$\Box$}
\checkedchar{$\blacksquare$}
\bracketedpoints
\shadedsolutions
\renewcommand{\solutiontitle}{\noindent\textbf{Solution:}\par\noindent}
\definecolor{SolutionColor}{gray}{0.8}
%\def\explanation#1{%
\def\explanation#1{%
  \ifprintanswers\mdseries\hspace{1em}\colorbox{SolutionColor}‌​{\strut #1}
  \ifprintanswers\mdseries\hspace{1em}\colorbox{SolutionColor}{\strut #1}
  \else 
  \enspace\hrulefill
  \fi}

\renewcommand{\solutiontitle}{\noindent\textbf{Solução:}\par\noindent}


\begin{document}

%Descomente a linha a seguir para imprimir as soluções.
\printanswers

\begin{center}
\begin{minipage}[c][1.5cm][c]{1.5cm}
\end{minipage}
\begin{minipage}[c][1.5cm][c]{12cm} 
\textsc{\Large Universidade Estadual de Maringá} \\
Departamento de Informática -- Projeto e Análise de Algoritmos \\
Prof. Daniel Kikuti
\end{minipage}
\end{center}

\begin{center}
\section*{Trabalho 1}
\end{center}

\bracketedpoints
\pointname{}
\pointformat{[Valor: \thepoints]}

\begin{questions}
  %\large
  %01
  \question Para cada par de funções $f(n)$ e $g(n)$ na tabela a
  seguir, indique se $f(n)$ pertence a $O(g(n))$, $\Omega(g(n))$
  ou $\Theta(g(n))$. Considere que $k \geq 1$, $\epsilon > 0$ e
  $c > 1$. Justifique sua resposta.
  \begin{center}
    \begin{tabular}{|c|c|c|c|c|c|}
      \hline
      & $f(n)$ & $g(n)$ & $f(n) = O(g(n))$? & $f(n) = \Omega(g(n))$? & $f(n) = \Theta(g(n))$? \\
      \hline
      a) & $\lg^k n$  &  $n^{\epsilon}$   & & X & \\ \hline
      b) & $n^k$      &  $c^n$          & X & & \\ \hline
      c) & $2^n$      &  $2^{n/2}$       & & X & \\ \hline
      d) & $n^{\lg c}$ &  $c^{\lg n}$      & & & X \\ \hline
    \end{tabular}
  \end{center}
  \begin{solution}
    Usando limite para analisar qual complexidade $f(n)$ corresponde em $g(n)$ aplicamos: 
    L =  $\displaystyle\lim_{n\to\infty} \frac{f(n)}{g(n)}$

    a) $\displaystyle\lim_{n\to\infty} \frac{\lg^k n}{n^{\epsilon}} = \displaystyle\lim_{n\to\infty} n^{-\epsilon}\lg^k n
      = \displaystyle\lim_{n\to\infty} n^{-\epsilon} . \displaystyle\lim_{n\to\infty} \lg^k n 
      = \infty \Rightarrow f(n) = \Omega(g(n))$ \\

    b) $\displaystyle\lim_{n\to\infty} \frac{n^k}{c^n} = \displaystyle\lim_{n\to\infty} \frac{\log_n n^k}{\log_n c^n}
      = \displaystyle\lim_{n\to\infty} \frac{k}{\log_n c^n} \\ \\
      Como \displaystyle\lim_{n\to\infty} \log_n c^n = \infty \\
       \\\displaystyle\lim_{n\to\infty} \frac{k}{\log_n c^n} = 0 \Rightarrow f(n) = O(g(n))$ \\

    c) $\displaystyle\lim_{n\to\infty} \frac{2^n}{2^{n/2}} = \displaystyle\lim_{n\to\infty} 2^n.{2^{-\frac{n}{2}}}
      = \displaystyle\lim_{n\to\infty} 2^{\frac{n}{2}} = \infty \Rightarrow f(n) = \Omega(g(n))$

    d) $\displaystyle\lim_{n\to\infty} \frac{n^{\lg c}}{c^{\lg n}} = \displaystyle\lim_{n\to\infty} \frac{n^{\lg c}}{n^{\lg c}}
      = \displaystyle\lim_{n\to\infty} 1 = 1 \Rightarrow f(n) = \Theta(g(n))$
  \end{solution}
  %02
  \question Para cada item a seguir, assinale \textbf{V}erdadeiro 
  ou \textbf{F}also. \textbf{Justifique} sua resposta usando as 
  definições de notação assintótica.
  \begin{parts}
    \part 
    \mbox{\begin{oneparcheckboxes}
      \correctchoice V \choice F
    \end{oneparcheckboxes}}
    Se $f(n) = \log_{16} n$ então $f(n) = \Theta(\lg n)$?
    \part 
    \mbox{\begin{oneparcheckboxes}
      \choice V \correctchoice F
    \end{oneparcheckboxes}}
    $2^{n + a} = \Theta(2^{2n})$? Onde $a \in \mathbb{N}$ é uma constante.
    \part
    \mbox{\begin{oneparcheckboxes}
      \correctchoice V \choice F
    \end{oneparcheckboxes}}
    $\frac{n^{2}}{4} - 3n - 16 = \Omega(n^2)$?
    \part
    \mbox{\begin{oneparcheckboxes}
      \correctchoice V \choice F
    \end{oneparcheckboxes}}
    $7n^2 + 13n = O(n^2)$.
  \end{parts}
  \begin{solution}
    a) $\displaystyle\lim_{n\to\infty} \frac{\log_{16} n}{\lg n} = \frac{1}{4} \to f(n) = \Theta(g(n))$ \\
    b) $\displaystyle\lim_{n\to\infty} \frac{2^{n+a}}{2^{2n}} = 0 \Rightarrow f(n) = 2^{n+a} \in O(2^{2n})$ \\
    c) $\displaystyle\lim_{n\to\infty} \frac{\frac{n^2}{4} -3n - 16}{n^2} = \frac{1}{4} \Rightarrow f(n) = \frac{n^2}{4} -3n - 16 
        \in \Theta(n^2) =  O(n^2) \land  \Omega(n^2)  \Rightarrow f(n) \in \Omega(n^2)$\\
    d) $\displaystyle\lim_{n\to\infty} \frac{7n^2 +13n}{n^2} = 7 \to f(n) = 7n^2 +13n
        \in \Theta(n^2) =  O(n^2) \land  \Omega(n^2) \Rightarrow f(n) \in O(n^2)$
  \end{solution}
  %03
  \question Mostre usando as definições de notação assintótica:
  \begin{parts}
    \part $\frac{n}{2}\lg(\frac{n}{2}) = \Omega(n \lg n)$.
    \part $n^2 + 1000n = O(n^2)$.
    \part $2^{n+1} = \Theta(2^{n})$.
  \end{parts}
  %04
  \question Expresse as seguintes funções em termos da notação $\Theta$.
  \begin{parts}
    \part $2n + 3 \log ^{100} n$.
    \part $7n^3 + 1000n \log n + 3n$.
    \part $3n^{1.5} + (\sqrt{n})^3 \log n$.
    \part $2^n + 100^n + n!$.
  \end{parts}
  %05
  \question É comum usar a notação $f(n) \prec g(n)$ para denotar que
  $f(n) \in o(g(n))$. Use esta notação para expressar a
  hierarquia de classes de complexidade das seguintes funções:
  $\sqrt{n}$, $2^{n^2}$, $n$, $\lg n$, $1$, $\lg \lg n$, $n!$, $n^2$, $n^{3/4}$, $2^n$, $n \lg n$.
  %06
  \question Sejam $f(n)$ e $g(n)$ funções positivas. Informe se a
  afirmação é verdadeira ou falsa e justifique.
  \begin{parts}
    \part $f(n) = O(g(n))$ implica $g(n) = O(f(n))$.
    \part $f(n) + g(n) = \Theta(min(f(n),g(n)))$
    \part $f(n) = O(g(n))$ implica $g(n) = \Omega(f(n))$
    \part $f(n) = \Theta(f(n/2))$
  \end{parts}
  %07
  \question Mostre uma função $f(n)$ tal que $f(n) \not\in \Omega(f(n+1))$.
  %08
  \question Mostre que $\sum_{i=1}^{n} \lg i = \Theta(n \lg n)$.
  %09
  \question Mostre que $n! = O(2^{n^2})$.
  %10
  \question Seja $p(n) = \sum_{i=0}^{k} a_{i}n^{i}$
  (polinômio em $n$ de grau $k$), onde $k$ é um inteiro não-negativo, $a_i$
  é uma constante e $a_k > 0$, mostre que $p(n) = \Theta(n^k)$.
  \begin{solution}
    Usando limite, devemos mostrar que: L = $\displaystyle\lim_{n\to\infty} \frac{\sum_{i=0}^{k} a_{i}n^{i}}{n^k}, 0 < L < \infty$\\
    \begin{tabular}{lll}
      $ L $&  $=$ &$ \displaystyle\lim_{n\to\infty} \frac{\sum_{i=0}^{k} a_{i}n^{i}}{n^k}$ \\
      {}& $=$ &$ \displaystyle\lim_{n\to\infty} \frac{a_0 n^0 + a_1 n^1 + a_2 n^2 + ... + a_k n^k}{n^k}$ \\
      {}& $=$ &$ \displaystyle\lim_{n\to\infty} \frac{a_0 n^0}{n^k} + 
                \displaystyle\lim_{n\to\infty} \frac{a_1 n^1}{n^k} + 
                \displaystyle\lim_{n\to\infty} \frac{a_2 n^2}{n^k} + ... + 
                \displaystyle\lim_{n\to\infty} \frac{a_k n^k}{n^k} $\\
      {}& $=$ &$ 0 + 0 + 0 + a_k $ \\
      {}& $=$ &$ a_k. $\\
    \end{tabular}
  \end{solution}
  %11
  \question Papai Noel resolveu antecipar seu presente de
  Natal. Sabendo que você foi um(a) bom(a) menino(a), ele escreveu um
  algoritmo para que você analise e ganhe uns pontinhos na prova de
  PAA. Sua tarefa é simples. Dado um inteiro $n$ como entrada,
  expresse por meio de notação assintótica a quantidade de ``Ho!''s
  que será impressa pelo algoritmo (use a notação $\Theta$).
  
  \begin{algorithm}[H]
    \NoCaptionOfAlgo
    \DontPrintSemicolon

    $i \leftarrow 1$\;
    \While{$i \leq n$} {
      \For {$j \leftarrow i$ \KwTo $2i - 1$}{
        print ``Ho!'';
      }
      $i \leftarrow 2i$\;
    }
    \caption{\textsc{Feliz-Natal}($n$)}
  \end{algorithm}
  %12
  \question Dado um inteiro $n$ (assuma que $n=2^k$, tal que
  $k$ é um número inteiro positivo) e $expr$ (que corresponde a uma
  expressão a ser impressa), informe, por meio de notação assintótica,
  a quantidade de mensagens que o algoritmo a seguir irá
  imprimir. Dê sua resposta em função de $n$.

  \begin{algorithm}[H]
    \NoCaptionOfAlgo
    \DontPrintSemicolon

    \While {$n \geq 1$}{
      \For {$j \gets 1$ \KwTo $n$}{
        print $expr$\;
      }
      $n \gets n / 2$
    }
    \caption{\textsc{Prog}($n$, $expr$)}
  \end{algorithm}
  \begin{solution}
    A cada repetição do While a linha 3 executará $\frac{n}{2}$ a cada iteração. O While executa $\lg n$ vezes,
    então a linha 3 será executada $\displaystyle\sum_{k=1}^{\lg n} \dfrac{n}{2^k}$ . Encontraremos a fórmula fechada
    para esse somatório:\\
    \begin{tabular}{lll}
      $ \displaystyle\sum_{k=1}^{\lg n} \dfrac{n}{2^k} $&  $=$ &$ n\displaystyle\sum_{k=1}^{\lg n} {\dfrac{1}{2}}^k $ \\
      {}& $=$ &$ n\frac{\frac{1}{2}({\frac{1}{2}}^{\lg n} - {\frac{1}{2}}^{-1})}{\frac{1}{2} - 1}$ \\
      {}& $=$ &$ n\frac{\frac{1}{2}({n}^{\lg {\frac{1}{2}}} - {2})}{-\frac{1}{2}}$ \\
      {}& $=$ &$ n\frac{\frac{1}{2}({n}^{\lg 1 - \lg 2} - {2})}{-\frac{1}{2}}$ \\
      {}& $=$ &$ -n({n^{{\lg 1} - {\lg 2}}} - 2)$ \\
      {}& $=$ &$ 2n -1$ \\
    \end{tabular}

    $\therefore$ O algorítimo irá imprimir a mensagem $2n-1$ vezes uma entrada $n$ e $T(n)= \Theta(n)$.
  \end{solution}
  %13
  \question Seja $count$ o número total de iterações feitas pelo
  algoritmo a seguir para uma entrada $n$. Informe, usando
  notação assintótica, o valor de $count$ em função de $n$.

  \begin{algorithm}[H]
    \NoCaptionOfAlgo
    \DontPrintSemicolon

    $count \gets 0$\;
    \For {$i \gets 1$ \KwTo $n$}{
      \For {$j \gets 1$ \KwTo $\floor{n/i}$}{
        $count \gets count + 1$\;
      }
    }
    \Return $count$\;
    \caption{\textsc{Count}($n$)}
  \end{algorithm}
  %14
  \question Seja $count$ o número total de iterações feitas pelo
  algoritmo a seguir para uma entrada $n$ (considere que $n = 2^{2^{k}}$,
  para algum inteiro positivo $k$). Informe, usando
  notação assintótica, o valor de $count$ em função de $n$.
  
  \begin{algorithm}[H]
    \NoCaptionOfAlgo
    \DontPrintSemicolon

    $count \gets 0$\;
    \For {$i \gets 1$ \KwTo $n$}{
      \For {$j \gets 2; j \leq n; j \gets j^2$}{
        $count \gets count + 1$\;
      }
    }
    \Return $count$\;
    \caption{\textsc{Count}($n$)}
  \end{algorithm}
  %15
  \question Dado um inteiro $n \geq 0$ como entrada, muitos
  afirmam que o algoritmo a seguir é capaz de medir o desespero na
  prova de PAA. Outros afirmam que o algoritmo mede a alegria. Para
  o professor, não interessa o que o algoritmo mede. O objetivo
  desta questão é avaliar se o aluno é capaz de encontrar uma
  fórmula fechada que representa o valor final de $x$ em função do
  valor de entrada $n$. Em outras palavras, uma função que
  representa quantas vezes a linha \ref{incrementa} será executada.

  \begin{algorithm}[H]
    \NoCaptionOfAlgo
    \DontPrintSemicolon

    $x \gets 0$\;
    \For {$i \gets 1$ \KwTo $n$}{
      \For {$j \gets i + 1$ \KwTo $n$}{
        \For {$k \gets 1$ \KwTo $j - i$}{
          $x \gets x + 1$ \label{incrementa}\;
        }
      }
    }
    \caption{\textsc{Prog}($n$)}
  \end{algorithm}
  \begin{solution}
    \begin{tabular}{lll}
      $ \displaystyle\sum_{k=1}^{n-1} \dfrac{(n-k+1)(n-k)}{2} $&  $=$ &$ \displaystyle\sum_{k=1}^{n-1} \dfrac{(n-k)^2 + (n-k)}{2} $\\
      {}& $=$ &$ \displaystyle\sum_{k=1}^{n-1} \dfrac{(n-k)^2}{2} + \displaystyle\sum_{k=1}^{n-1}  \dfrac{n}{2} - \displaystyle\sum_{k=1}^{n-1} \dfrac{k}{2} $ \\
      {}& $=$ &$ \dfrac{1}{2}\left[ \displaystyle\sum_{k=1}^{n-1} k ^2 \right] + \dfrac{1}{2} \left[ n(n-1) - \dfrac{n(n-1)}{2} \right]  $\\ 
      {}& $=$ &$ \dfrac{1}{2} \left[ \displaystyle\sum_{k=1}^{n-1} k^2 \right] + \dfrac{n(n-1)}{4} $\\
      {}& $=$ &$ \dfrac{1}{2} \left[ \displaystyle\sum_{k=1}^{n-1} k^2 \right] + \dfrac{n^2 -n}{4} $\\
    \end{tabular}

      Como $\displaystyle\sum_{k=1}^{n-1} k^2 $ é a soma das $n-1$  primeiras linha do triângulo de pascal, podemos concluir

    \begin{tabular}{lll}
      $\displaystyle\sum_{k=1}^{n-1} \dfrac{(n-k+1)(n-k)}{2}$& $=$ &$\dfrac{1}{2} \left[ \dfrac{n(n-1)(2n-1)}{6}\right] + \dfrac{n^2 + n}{4} $\\
      {}& $=$ &$ \dfrac{(n^2 -n)(2n-1)}{12} + \dfrac{n^2 - n}{4}  $\\
      {}& $=$ &$ \dfrac{2n^3 - 3n^2 + n}{12} + \dfrac{3n^2 - 3n}{12}$\\
      {}& $=$ &$ \dfrac{2n^3- 2n}{12} $\\
      {}& $=$ &$ \dfrac{n^3 -n}{6} $\\
    \end{tabular}
  \end{solution}
  %16
  \question Resolva as seguintes recorrências (use o método da
  substituição):
  \begin{parts}
    \part $T(n) = 8T(n/2) + \Theta(n^2)$
    \part $T(n) = 7T(n/2) + \Theta(n^2)$
    \part $T(n) = T(n/4) + 1$
    \part $T(n) = 2T(n/2) + n \lg n$
  \end{parts}
  \begin{solution}
    \textbf{a)$T(n) = 8T(n/2) + \Theta(n^2)$}\\

    Supondo que $T(n) \in O(n^3 - n^2)$ e substituindo em $T(n):$\\
    \begin{tabular}{lll}
      $T(n) $& $=$ &$ 8T(n/2) + \Theta(n^2)$ \\
      {}& $=$ &$ 8c(\frac{n^3}{8} - \frac{n^2}{4}) + \Theta(n^2)$\\
      {}& $=$ &$ c.{n^3} - 2.{n^2} + d.n^2$ \\
      {}& $=$ &$ c.{n^3} - n^2({2c -d})$\\
    \end{tabular}

    $p/ c \geq \frac{d}{2} \Rightarrow T(n) = O(n^3)$\\

    \textbf{b)$T(n) = 7T(n/2) + \Theta(n^2)$}\\

    Supondo que $T(n) \in O(n^{\lg 7} - n^2)$ e substituindo em $T(n):$\\
    \begin{tabular}{lll}
      $T(n) $& $=$ &$ 7T(n/2) + \Theta(n^2)$ \\
      {}& $=$ &$ 7c({(\frac{n}{2}})^{\lg 7} - \frac{n^2}{4}) + \Theta(n^2)$\\
      {}& $=$ &$ 7c({7^{\lg \frac{n}{2}} - \frac{n^2}{4}}) + \Theta(n^2)$\\
      {}& $=$ &$ c{7^{\lg n} - \frac{7n^2}{4}} + \Theta(n^2)$\\
      {}& $=$ &$ c{7^{\lg n} - n^2(\frac{7}{4}} - d)$\\
    \end{tabular}

    $p/ c \geq \frac{7}{4} \Rightarrow T(n) = O(n^{\lg 7})$\\

    \textbf{c)$T(n) = T(n/4) + 1$}\\

    Supondo que $T(n) \in O(\lg n)$ e substituindo em $T(n):$\\
    \begin{tabular}{lll}
      $T(n) $& $=$ &$ T(n) = T(n/4) + 1$ \\
      {}& $=$ &$ c(\frac{1}{4} {\lg n} + 1$\\
      {}& $=$ &$ c(\frac{1}{4} {\lg n} + 1$\\
      {}& $=$ &$ \frac{1}{4}c{\lg n} + 1$\\
    \end{tabular}

    $p/ c > 4 \Rightarrow T(n) = O(n^{\lg n})$\\

    \textbf{d)$T(n) = 2T(n/2) + n\lg n$}\\

    Supondo que $T(n) \in O(n\lg^2 n)$ e substituindo em $T(n):$\\
    \begin{tabular}{lll}
      $T(n) $& $=$ &$ T(n) = 2T(n/2) + n\lg n$ \\
      {}& $=$ &$ 2c(\frac{n}{2} \lg^2 n) + n\lg n$\\
      {}& $=$ &$ cn\lg^2 n + n\lg n$\\
    \end{tabular}

    $p/ c > 0 \Rightarrow T(n) = O(n\lg^2 n)$\\    
  \end{solution}
  %17
  \question Use árvore de recorrência para estimar um limite superior
  para as seguintes recorrências. Assuma que $T(n)$ é uma constante
  para $n \leq 2$. Depois comprove usando o método de
  substituição.
  \begin{parts}
    \part $T(n/2) + T(n/4) + T(n/8) +n$
    \part $2T(n/4) + \sqrt{n}$
  \end{parts}
  %18
  \question Utilize o método de árvore de recursão para supor um
  limite assintótico superior para a recorrência \linebreak
  $T(n) = 3T(n-1) + 1$. Depois verifique pelo método de
  substituição que este limite está correto.
  %19
  \question Utilize o método de árvore de recursão para supor um
  limite assintótico superior para a recorrência \linebreak
  $T(n) = 2T(n/2) + n \lg n$. Depois verifique pelo método de
  substituição que este limite está correto. 
  %20
  \question Utilize o método de árvore de recursão para supor um
  limite assintótico superior para a recorrência \linebreak $T(n) = T(n/3) +
  T(2n/3) + \Theta(n)$. Depois verifique pelo método de
  substituição que este limite está correto.
  %21
  \question A recorrência $T(n) = 7T(n/2) + n^2$ descreve o
  tempo de execução de um algoritmo $A$. Um algoritmo alternativo $A'$ 
  tem um tempo de execução $T'(n) = aT'(n/4) + n^2$. Qual é o maior
  inteiro $a$ que faz com que $A'$ seja assintoticamente mais rápido
  que $A$?
  %22
  \question Use o método mestre para resolver as seguintes
  recorrências: 
  \begin{parts}
    \part $T(n) = 3T(n/2) + n \lg n$ 
    \part $T(n) = 3T(n/2) + n^2$
    \part $T(n) = 4T(n/2) + n^2$
    \part $T(n) = 4T(n/2) + n^2\sqrt{n}$
    \part $T(n) = 5T(n/5) + n$  
    \part $T(n) = 6T(n/3) + n^2$
    \part $T(n) = 9T(n/2) + n^3$
  \end{parts}
  \begin{solution}
    \textbf{a)$T(n) = 3T(n/2) + n\lg n$}\\
      $\log_b a = \lg 3 \simeq 1.58$ e $f(n) = n\lg n$ $\Rightarrow$  Caso 1 do Método Mestre\\
      $f(n) \in O(n^{\lg 3 - \epsilon})$, com $\epsilon = 0.08 \Rightarrow f(n) \in O(n^{\frac{3}{2}})$
      $\Rightarrow T(n) = \Theta(n^{\lg 3})$\\

    \textbf{b)$T(n) = 3T(n/2) + n^2$}\\
      $\log_b a = \lg 3 \simeq 1.58$ e $f(n) = n^2$ $\Rightarrow$  Caso 3 do Método Mestre\\
      $f(n) \in O(n^{\lg 3 + \epsilon})$, com $\epsilon = 0.42$
      $\Rightarrow T(n) = \Theta(f(n)) = \Theta(n^2)$\\

    \textbf{c)$T(n) = 4T(n/2) + n^2$}\\
      $\log_b a = 2 $ e $f(n) = n^2$ $\Rightarrow$  Caso 2 do Método Mestre\\
      $f(n) \in \Theta(n^2)$
      $\Rightarrow T(n) = \Theta(n^2 \lg n)$\\

    \textbf{d)$T(n) = 4T(n/2) + n^2\sqrt{n}$}\\
      $\log_b a = 2 $ e $f(n) = n^2.\sqrt{n} = n^{\frac{3}{2}}$ $\Rightarrow$  Caso 3 do Método Mestre\\
      $f(n) \in \Theta(n^{2 + \epsilon})$, com $\epsilon = 0.5$
      $\Rightarrow T(n) = \Theta(f(n)) = \Theta(n^2.\sqrt{n})$\\

    \textbf{e)$T(n) = 5T(n/5) + n$}\\
      $\log_b a = 1 $ e $f(n) = n$ $\Rightarrow$  Caso 2 do Método Mestre\\
      $f(n) \in \Theta(n)$
      $\Rightarrow T(n) = \Theta(n\lg n)$\\

    \textbf{f)$T(n) = 6T(n/3) + n^2$}\\
      $\log_b a = \log_3 6 \simeq 1.63$ e $f(n) = n^2$ $\Rightarrow$  Caso 3 do Método Mestre\\
      $f(n) \in \Omega(n^{\log_3 6 + \epsilon})$, com $\epsilon = 0.33$
      $\Rightarrow T(n) = \Theta(f(n)) = \Theta(n^2)$\\
      
    \textbf{g)$T(n) = 9T(n/2) + n^3$}\\
      $\log_b a = \lg 9 \simeq 3.16$ e $f(n) = n^3$ $\Rightarrow$  Caso 1 do Método Mestre\\
      $f(n) \in O(n^{\lg 9 - \epsilon})$, com $\epsilon = 0.16 \Rightarrow f(n) \in O(n^3)$
      $\Rightarrow T(n) = \Theta(n^{\lg 3})$
  \end{solution}
  %23
  \question Dadas as recorrências dos algoritmos $A$ e
  $B$, determine a complexidade de cada um deles e compare-os
  (informe se $A$ é assintoticamente mais rápido que $B$, $B$
  é assintoticamente mais rápido que $A$, ou ambos possuem a
  mesma complexidade assintótica).
  \begin{itemize}
  \item $T_A(n) = 27T_A(n/3) + n$
  \item $T_B(n) = 4T_B(n/2) + n^3$
  \end{itemize}
  %24
  \question Os três algoritmos a seguir resolvem um
  problema de tamanho $n$ por meio da técnica de divisão e conquista.
  Analise a complexidade de cada um deles e informe qual algoritmo é
  assintoticamente mais eficiente.
  \begin{itemize}
  \item Algoritmo $A$ resolve problemas dividindo-os em cinco
    subproblemas de tamanho $n/2$, recursivamente resolve cada
    subproblema e combina suas soluções em tempo linear para obter
    uma solução do problema original.
    \item Algoritmo $B$ resolve problemas dividindo-os em dois
      subproblemas de tamanho $n-1$, recursivamente resolve cada um
      dos subproblemas e combina as soluções em tempo constante para
      obter a solução do problema original.
    \item Algoritmo $C$ resolve problemas dividindo-os em nove 
      subproblemas de tamanho $n/3$, recursivamente resolve cada
      subproblema e combina suas soluções em tempo $O(n^2)$ para obter
      uma solução do problema original.
  \end{itemize}
  %25
  \question Os três algoritmos a seguir computam corretamente
  $x^n$ para $x>0$ e $n \geq 0$. Mostre que os três algoritmos estão
  corretos e analise a complexidade assintótica
  de cada um deles (em função de $n$) e informe qual deles é mais
  eficiente.
  \setlength{\algoheightrule}{0pt}
  \setlength{\algotitleheightrule}{0pt}
  \begin{center}
    \vspace{-1cm}
    \begin{minipage}[t][][c]{.25\textwidth}
      \begin{algorithm}[H]
        \NoCaptionOfAlgo
        \DontPrintSemicolon
        $resp \gets 1$\;
        $i \gets 0$\;
        \While {$i < n$} {
          $resp \gets resp * x$\;
          $i \gets i + 1$\;
        }
        \Return $resp$\;
        \caption{\textsc{Power1}($x, n$)}
      \end{algorithm}
    \end{minipage}%
  \begin{minipage}[t][][c]{.35\textwidth} 
      \begin{algorithm}[H]
        \NoCaptionOfAlgo
        \DontPrintSemicolon
        \If {$n = 0$}{
          \Return $1$\;
        }
        \Else {
            \Return \textsc{Power2}($x, n-1$) $*$ $x$\;
        }
        \caption{\textsc{Power2}($x,n$)}
      \end{algorithm}
    \end{minipage}%
    \begin{minipage}[t][][c]{.38\textwidth} 
      \begin{algorithm}[H]
        \NoCaptionOfAlgo
        \DontPrintSemicolon
        \If {$n = 0$}{
          \Return $1$\;
        }
        \ElseIf {($n$ mod $2$) = 0} {
          $aux \leftarrow $ \textsc{Power3}($x, n/2$)\;
          \Return $aux*aux$\;
        }
        \Else {
          \Return \textsc{Power3}($x, n-1$) $*$ $x$\;
        }
        \caption{\textsc{Power3}($x,n$)}
      \end{algorithm}
    \end{minipage}
  \end{center}
  %26
  \question Considere o algoritmo \textsc{Heapsort} descrito a
  seguir. Mostre que o algoritmo está correto usando o seguinte
  invariante de laço: ``No começo de cada iteração do laço
  \textbf{for} das linhas 2--5, o subvetor $A[1 \dots i]$ contém os $i$
  menores elementos de $A[1 \dots n]$, e o subvetor $A[i{+}1 \dots n]$ contém
  os $n-i$ maiores elementos de $A[1 \dots n]$ em ordem.

  \setlength{\algoheightrule}{0pt}
  \setlength{\algotitleheightrule}{0pt}
  
  \IncMargin{1em}
  \begin{algorithm}[H]
    \NoCaptionOfAlgo
    \DontPrintSemicolon

    \textsc{Build-Max-Heap}($A$)\;
    \For {$i \leftarrow A.length$ \KwTo $2$}{
      \textsc{swap}($A[1], A[i]$)\;
      $A.heap$-$size \leftarrow A.heap$-$size - 1$\;
      \textsc{Max-Heapify}($A, 1$)\;
    }

    \caption{\textsc{Heapsort}($A$)}
  \end{algorithm}
  \DecMargin{1em}
  %27
  \question Analise a complexidade dos seguintes algoritmos:
  \begin{parts}
    \part
    \begin{algorithm}[H]
      \NoCaptionOfAlgo
      \DontPrintSemicolon
      \If {$n = 1$}{
        \Return 1\;
      }
      \Else {
        \Return \textsc{f}($n - 1$) + \textsc{f}($n - 1$)\;
      }
      \caption{\textsc{f}($n$)}
    \end{algorithm}
    \begin{solution}
      \begin{tabular}{lll}
        $2T(n-1) + \Theta(1)$ $=$
        $\displaystyle\sum_{k=0}^{\lg n - 1} 2^k + \Theta(1)$
        {}& $=$ &$ \frac{2^{n+1-1} - 1}{2 - 1} + \Theta(1)$ \\
        {}& $=$ &$ 2^{n} -1 + \Theta(1)$ \\
        {}& $=$ &$ O(2^n)$ \\
      \end{tabular}  
    \end{solution}
    
    \part
    \begin{algorithm}[H]
      \NoCaptionOfAlgo
      \DontPrintSemicolon
      \If {$max < min$}{
        \Return -1\;
      }
      $mid \leftarrow min + ((max - min)/2)$\;
      \If {$A[mid] > key$}{
        \Return \textsc{Busca}($A, key, min, mid-1$)\;
      }
      \ElseIf {A[mid] < key} {
        \Return \textsc{Busca}($A, key, mid+1, max$)\;
      }
      \Else {
        \Return $mid$\;
      }
      \caption{\textsc{Busca}($A[], key, min, max$)}
    \end{algorithm}
    \begin{solution}
      \begin{tabular}{lll}
        $T(\frac{n}{2}) + \Theta(1)$ $=$
        $\displaystyle\sum_{k=0}^{\lg n - 1} 1 + \Theta(1)$
        {}& $=$ &$ \lg n - 1 + \Theta(1)$ \\
        {}& $=$ &$ O(\lg n)$ \\
      \end{tabular}  
    \end{solution}
  
    \part
    \begin{algorithm}[H]
      \NoCaptionOfAlgo
      \DontPrintSemicolon
      \If {$n > 1$}{
        \For{$i \leftarrow 1$ \KwTo $n^3$}{
          Faz algo (custo $\Theta(1)$)\;
        }
        \textsc{Recursive}($n/3$)\;
      }
      \caption{\textsc{Recursive}($n$)}
    \end{algorithm}
    \begin{solution}
      \begin{tabular}{lll}
        $T(\frac{n}{3}) + \Theta(n^3)$ $=$
        $n^3 \displaystyle\sum_{k=0}^{n - 1} {\frac{1}{27}}^k + \Theta(n^3)$
        {}& $=$ &$ n^3 \frac{1}{1 - \frac{1}{27}} + \Theta(n^3)$ \\
        {}& $=$ &$ n^3 \frac{26}{27} + \Theta(n^3)$ \\
        {}& $=$ &$ O(n^3)$ \\
      \end{tabular}
    \end{solution}

  \end{parts}
  %28
  \question Assinale Verdadeiro (V) ou Falso (F). \textbf{Justifique}
  \begin{parts}
    \part \mbox{\begin{oneparcheckboxes}
        \correctchoice V \choice F
    \end{oneparcheckboxes}}
    O limite assintótico inferior para algoritmos de ordenação
    baseados em comparação é $\Omega(n \lg n)$. Um algoritmo de
    ordenação por comparação que faz $2T(n/2) + \Theta(1)$
    comparações no pior caso com certeza não efetua corretamente
    a ordenação para algumas instâncias.
    
    \part \mbox{\begin{oneparcheckboxes}
      \choice V \correctchoice F
    \end{oneparcheckboxes}}
    Suponha que iremos gerar $n$ números aleatórios no intervalo
    $[0 \ldots n^2]$. Considerando base decimal (isto é, um número $x$
    possui $\lfloor \log_{10} x \rfloor + 1$ dígitos na base $10$), é
    correto afirmar que o algoritmo \textsc{Radix Sort}
    faz a ordenação destes $n$ números em tempo $O(n)$.

    \part \mbox{\begin{oneparcheckboxes}
        \choice V \correctchoice F
    \end{oneparcheckboxes}}
    Visto que o limite assintótico inferior para algoritmos de ordenação
    baseados em comparação é $\Omega(n \lg n)$, não seria possível o
    desenvolvimento de um algoritmo de ordenação correto com complexidade
    de tempo $O(n \sqrt{n})$ no pior caso.
    
    \part \mbox{\begin{oneparcheckboxes}
        \choice V \correctchoice F
    \end{oneparcheckboxes}}
    Suponha que iremos gerar $n$ números aleatórios no intervalo
    $[0 \ldots n^2]$. É correto afirmar que o algoritmo \textsc{Counting Sort}
    faz a ordenação destes $n$ números em tempo $O(n)$.
    
    \part \mbox{\begin{oneparcheckboxes}
        \correctchoice V \choice F
    \end{oneparcheckboxes}}
    Suponha que iremos gerar $n$ números aleatórios no intervalo
    $[0 \ldots n^2]$. É correto afirmar que o algoritmo \textsc{Mergesort}
    faz a ordenação destes $n$ números em tempo $O(n \lg n)$.

    \part \mbox{\begin{oneparcheckboxes}
        \choice V \correctchoice F
    \end{oneparcheckboxes}}
    Não é possível construir um heap máximo com $n$ elementos em tempo
    $O(n)$. Pois para inserir um elemento no heap temos custo $O(\lg n)$ e,
    como temos $n$ elementos a serem inseridos, o custo total seria pelo
    menos $O(n \lg n)$.

    \part \mbox{\begin{oneparcheckboxes}
      \correctchoice V \choice F
    \end{oneparcheckboxes}}
    Em um heap binário, metade dos elementos do vetor são
    folhas. Se aplicarmos o procedimento \textsc{MaxHeapfy}
    para cada elemento da metade até o primeiro, então 
    ao fim teremos um Heap Máximo.

    \part \mbox{\begin{oneparcheckboxes}
        \correctchoice V \choice F
    \end{oneparcheckboxes}}
    É correto afirmar que: no melhor caso, o algoritmo \textsc{Insertion Sort}
    é mais eficiente que os algoritmos \textsc{Mergesort} e \textsc{Heapsort}. 
  \end{parts}
  %29
  \question Use o modelo de árvore de decisão para representar as
  comparações efetuadas pelo algoritmo \textsc{Insertion-Sort} para
  uma instância de entrada com quatro elementos.
  %30
  \question Use o modelo de árvore de decisão para representar as
  comparações efetuadas pelo algoritmo \textsc{Mergesort} para uma instância
  de entrada com três elementos.
  \begin{solution}
        \begin{figure}[H]
          \begin{tikzpicture}[scale=0.68, transform shape]
            \node [intg] (kp)  {$a_2 : a_3$};
            \node [int] (ki1) [node distance=1.5cm and -1cm,below left=of kp] {$a_2 : a_1$};
            \node [int] (ki2) [node distance=1.5cm and -1cm,below right=of kp] {$a_1 : a_2$};
            \node [int] (ki3) [node distance=2cm,below left=of ki1] {$a_1 : a_3$};
            \node [int] (ki7) [node distance=1.5cm and -1cm,below right=of ki1] {$a_1,a_2,a_3$};
            \node [int] (ki4) [node distance=1.5cm and -1cm,below right=of ki2] {$a_1 : a_3$};
            \node [int] (ki9) [node distance=1.5cm and -1cm,below left=of ki4] {$a_2,a_3,a_1$};
            \node [int] (ki10) [node distance=1.5cm and -1cm,below right=of ki4] {$a_3,a_2,a_1$};
            \node [int] (ki5) [node distance=1.5cm and -1cm,below left=of ki3] {$a_2,a_1,a_3$};
            \node [int] (ki6) [node distance=1.5cm and -1cm,below right=of ki3] {$a_3,a_2,a_1$};
            \node [int] (ki8) [node distance=2.5cm and -1cm,below left=of ki2] {$a_1,a_3,a_2$};
            \draw[->] (kp) -- ($(kp.south)+(0,-0.75)$) -| (ki1) node[above,pos=0.25] {$\leq$} ;
            \draw[->] (ki1) -- ($(ki1.south)+(0,-0.75)$) -| (ki3) node[above,pos=0.25] {$\leq$} ;;
            \draw[->] (ki1) -- ($(ki1.south)+(0,-0.75)$) -| (ki7) node[above,pos=0.25] {$>$};
            \draw[->] (ki3) -- ($(ki3.south)+(0,-0.75)$) -| (ki5) node[above,pos=0.25] {$\leq$};
            \draw[->] (ki3) -- ($(ki3.south)+(0,-0.75)$) -| (ki6) node[above,pos=0.25] {$>$};
            \draw[->] (kp) -- ($(kp.south)+(0,-0.75)$) -| (ki2) node[above,pos=0.25] {$>$};
            \draw[->] (ki2) -- ($(ki2.south)+(0,-0.75)$) -| (ki4) node[above,pos=0.25] {$>$};
            \draw[->] (ki2) -- ($(ki2.south)+(0,-0.75)$) -| (ki8) node[above,pos=0.25] {$\leq$};
            \draw[->] (ki4) -- ($(ki4.south)+(0,-0.75)$) -| (ki9) node[above,pos=0.25] {$\leq$};
            \draw[->] (ki4) -- ($(ki4.south)+(0,-0.75)$) -| (ki10) node[above,pos=0.25] {$>$};
          \end{tikzpicture}
        \end{figure}
  \end{solution}
  %31
  \question Use o modelo de árvore de decisão para representar as
  comparações efetuadas pelo algoritmo \textsc{Quicksort} para uma
  instância de entrada com três elementos.
  %32
  \question Dado um vetor de inteiros distintos e ordenados em
  ordem crescente $A = \{a_1, \ldots, a_n\}$:
  \begin{parts}
    \part Descreva um algoritmo que determina se existe um
    índice $i$ tal que $a_i = i$ em tempo $O(\lg n)$. Por exemplo, em
    $\{-10, -3, 3, 5, 7\}$, $a_3 = 3$. Em $\{2,3,4,5,6,7\}$ não existe
    tal $i$. Argumente que seu algoritmo está correto.
    \part Explique por que sua solução leva tempo $O(\lg n)$.
  \end{parts}
  \begin{solution}
    \textbf{a)}
    \begin{algorithm}[H]
      \NoCaptionOfAlgo
      \DontPrintSemicolon
      \If {$inicio < fim$}{
        \Return $false$\;
      }
      $meio \leftarrow inicio + ((fim - inicio)/2)$\;
      \If{$A[meio] < meio$}{
        \Return $\textsc{BuscaC}($A[\quad], meio + 1, fim$)$\;
      }
      \ElseIf{$A[meio] > meio$}{
        \Return $\textsc{BuscaC}($A[\quad], inicio, meio - 1$)$\;
      }
      \Else{
        \Return $true$\;
      }
      \caption{\textsc{BuscaC}($A[\quad], inicio, fim$)}
    \end{algorithm}
    \textbf{b)} O algoritimo divide sucessivamente o vetor por dois sucessivamente e busca se o elemento $a_i = A[i]$, cada uma dessas operações
    tem tempo constante  e como o número de divisões é no máximo $\lg n$, temos o seguinte somatório:\\
    $\displaystyle\sum_{k=0}^{\lg n - 1} 1 + = \lg n \Rightarrow T(n) = \Theta(\lg n)$
  \end{solution}

  %33
  \question Descreva um algoritmo baseado no paradigma de Divisão
  e Conquista que encontra o mínimo de um conjunto de $n$ números.
  Assuma que os elementos estão em um vetor $A = [1\ldots n]$.
  Mostre que o algoritmo está correto e analise sua complexidade.
  \begin{solution}
    \begin{algorithm}[H]
      \NoCaptionOfAlgo
      \DontPrintSemicolon
      \If {$inicio = fim$}{
        \Return A[inicio]\;
      }\Else{
        $meio \leftarrow inicio + ((fim - inicio)/2)$\;
        $left = $\textsc{MinA}($A[\quad], inicio, meio$)\;
        $right = $\textsc{MinA}($A[\quad], meio + 1, fim$)\;
        \If{$left \geq right$}{
          \Return $right$\;
        }
        \Else{
          \Return $left$\;
        }
      }
      \caption{\textsc{MinA}($A[\quad], inicio, fim$)}
    \end{algorithm}
      $T(n) = 2T(n/2) + \Theta(1)$\\

      Por árvore de recursão temos que o custo por nível é igual a $n$, sabendo que a árvore tem tamanho $\lg n$ 
      e o custo das folhas é igual a $\Theta(n)$, e, fazendo o somatório concluimos que o custo total é ${n(\lg n - 1) + \Theta(n)}$ \\
      $\therefore T(n) = O(n\lg n)$\\


    \quad Para $n=1$ o elemento do inicio é trivialmente igual o do fim e é o menor elemento do vetor. Para um vetor de tamanho $k \geq 1$ o algorítimo divide no meio o vetor(linha 4), passa a metade da esquerda para $left$ e a metade da direita para $right$, que executarão a função de forma recuriva até chegar no caso base(n = 1), ou seja, vão de $n \ldots 1 $ obtendo os menores elementos da esquerda e da direita para cada chamada. As linhas 7 a 10 retornaram qual o menor elemento entre $left$  e $right$ de cada chamada recursiva, ou seja, ao final de cada chamada teremos o menor elemento de $\frac{n}{2}$. 
  \end{solution}
  %34
  \question Descreva um algoritmo baseado no paradigma de Divisão
  e Conquista que encontra o segundo maior elemento de um conjunto
  de $n$ números. Assuma que os elementos estão
  em um vetor $A = [1\ldots n]$. Mostre que o algoritmo está
  correto e analise sua complexidade.
  %35
  \question Descreva um algoritmo que faz uso do procedimento
  \textsc{Partition} para encontrar o $k$-ésimo menor elemento.
  Isto é, o algoritmo recebe como entrada um vetor $A[1 \ldots n]$ e um
  um valor $1 \leq k \leq n$ e devolve qual seria este $k$-ésimo
  menor elemento. Por exemplo, se $k = 1$ o algoritmo deveria
  devolver o mínimo do vetor; se $k=n$ o algoritmo deveria devolver
  o máximo; para um $k=3$ devolveria o terceiro menor elemento.
  Analise seu algoritmo no pior e no melhor caso.
  %36
  \question Assuma que você possui $k$ vetores ordenados, cada um
  com $n$ elementos, e você precisa combiná-los em um único vetor
  ordenado com $k n$ elementos.
  \begin{parts}
  \part Usando o procedimento \textsc{Merge}, faça a intercalação
  do primeiro vetor com o segundo, então intercale o terceiro, depois
  o quarto e assim por diante. Qual a complexidade de tempo deste
  algoritmo, em função de $k$ e $n$?
  \part Apresente uma solução mais eficiente para este problema, por
  meio da técnica de Divisão e Conquista. Qual a complexidade de
  tempo de sua solução, em função de $k$ e $n$?
  \end{parts}
  %37
  \question Dado um vetor de números inteiros $A[1 \ldots n]$, determine
  quais elementos do vetor são únicos. Apresente um algoritmo eficiente.
  Faça uma análise de complexidade.
  \begin{solution}
    Escreva a solução aqui.
  \end{solution}
  %38
  \question Descreva um algoritmo de tempo $\Theta(n \lg n)$ que, dado
  um conjunto $S$ de $n$ números inteiros e outro número $x$, determine
  se existe dois elementos em $S$ cuja soma é exatamente $x$.
  %39
  \question Problema da moeda falsa. Dado um conjunto de $n$ moedas,
  $n-1$ delas verdadeiras (com mesmo peso) e uma falsa (mais leve),
  descreva um algoritmo eficiente (com tempo $o(n)$) para encontrar
  a moeda falsa. 
\end{questions}
\end{document}